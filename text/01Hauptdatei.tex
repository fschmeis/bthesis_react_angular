%-----------------------------------------------%
%   Vorlage Abschlussarbeit HTWK Leipzig        %
%   Fakultät Maschinenbau und Energietechnik    %
%   Anpassungen/Support/Weiteres auf Anfrage    %
%                                               %
%        Der Code wird unter der Lizenz         %
%   „Creative Commons Namensnennung-Weitergabe  %
%   unter gleichen Bedingungen“ in Version 4.0  % 
%   (abgekürzt „CC-by-sa 4.0“) veröffentlicht.  %
%   _________________________________________   %
%-----------------------------------------------%

% Dokumentenklasse %
\documentclass[
ngerman,
%toc=flat,
toc=chapterentrywithdots,
captions=tableabove,
listof=entryprefix,
listof=leveldown,
fontsize=12pt,
numbers=noenddot,
parskip=half,
headsepline,
footsepline,
footheight=25pt,
headheight=25pt,
usegeometry,
]{scrreprt}

% Laden der Pakete %
\usepackage{babel}
\usepackage{lmodern}
\usepackage[T1]{fontenc}
\usepackage[utf8]{inputenc}

\usepackage{geometry}
\geometry{
	includeheadfoot,
	top = 20mm,
	left = 25mm,
	right = 20mm, 
	bottom = 30mm,
	bindingoffset = 5mm,
	}

\usepackage[letterspace=150]{microtype}
\usepackage[onehalfspacing]{setspace}
\BeforeStartingTOC[toc]{\singlespacing} 
\BeforeStartingTOC[lot]{\singlespacing\renewcommand\autodot{:}}
\BeforeStartingTOC[lof]{\singlespacing\renewcommand\autodot{:}}

\usepackage[
    backend=biber,
    style=numeric-comp,
    sorting=none,
    defernumbers=true
]{biblatex}
\addbibresource{Hilfsdateien/literatur.bib}
\usepackage{csquotes}

\usepackage{scrlayer-scrpage}
\clearscrheadfoot
\chead{{--~}\pagemark{~--}}

\renewcommand*{\chapterpagestyle}{scrheadings}
\RedeclareSectionCommand[%
  beforeskip=0pt
]{chapter}
\RedeclareSectionCommands[%
  beforeskip=0pt,
  afterskip=1pt plus .1\baselineskip minus .167\baselineskip
]{section,subsection}

\usepackage{graphicx} 
\usepackage{pdfpages}
\usepackage{siunitx}
\sisetup{
    locale=DE,
    per-mode=fraction,
}

\input{Hilfsdateien/06Anhangsverzeichnis}

\usepackage[
    pdfauthor=Felix Maximilian Schmeißer,
    pdfsubject=Vergleichende Analyse der komponentenbasierten Frontend-Frameworks Angular und React,
    plainpages=false
]{hyperref}

% Platz für weitere Pakete %
\usepackage{verbatim}
\usepackage{tikz}
\usepackage{dirtree}
\usepackage{forest}

% Sonstige Inputs
\usepackage{listings}
\usepackage{xcolor}
\usepackage[scaled]{DejaVuSansMono}

\usepackage{tcolorbox}
\newtcbox{\inlinecode}{on line, boxrule=0pt, boxsep=0pt, top=2pt, left=2pt, bottom=2pt, right=2pt, colback=gray!15, colframe=white, fontupper={\ttfamily \footnotesize}}


\usepackage{caption}
\usepackage[newfloat]{minted}

\DeclareCaptionFormat{captionFormat}{%
  \colorbox{gray!50}{\parbox{\dimexpr\textwidth-2\fboxsep-2\fboxrule\relax}{#1#2#3}}
} 
\captionsetup[listing]{position=top, skip=-10pt, format=captionFormat}

\setminted{frame=lines, framesep=2mm, numbersep=5pt, baselinestretch=1.2, fontsize=\footnotesize, linenos}


% Sonstiges
\newcommand*{\quelle}[1]{\par\raggedleft\footnotesize Quelle:~#1}

% Beginn des Dokumentes %
\begin{document}
\newpage
\thispagestyle{empty}
\quad
\addtocounter{page}{-1}
\newpage
%
\begin{titlepage}
{\centering
\includegraphics[]{Hilfsdateien/HTWK_logo.eps}\\[8ex]
{\large \textbf{Fakultät Informatik und Medien} \par}
{\large Studiengang Medieninformatik \par}
\vspace{1.75cm}
{\Large \textbf{Komponentenbasierte Frontend-Frameworks - Angular und React im Vergleich}\par}
\vspace{1.25cm}
{Bachelorarbeit \par}
{zur Erlangung des akademischen Grades \par} 
{\large  \textbf{Bachelor of Science} \par}
\vspace{1.25cm}
{\large vorgelegt von \par}
\vspace{1.25cm}
{Felix Schmeißer\\[3ex]
geb. am 22.\,02.\,1999\\[3ex]
in München\\[3ex]
69578 \par}}
\vfill
{\noindent Verantwortlicher Hochschullehrer: Prof. Dr. rer. nat. Klaus Hering\\
Leipzig, Juni 2020 -- September 2020}
\end{titlepage}


\thispagestyle{empty}
\begin{center}
\large \lsstyle Erklärung
\end{center}
\vspace{1.5cm}
{\doublespacing
Ich versichere wahrheitsgemäß, diese Arbeit selbständig angefertigt, alle benutzten Hilfsmittel vollständig und genau angegeben und alles kenntlich gemacht zu haben, was aus Arbeiten anderer unverändert oder mit Abänderungen entnommen wurde.\par
\vspace{2cm}
\noindent
\begin{minipage}[t]{6.5cm}
% gepunktete Linie
\dotfill \\
% Text unter der Linie
\onehalfspacing
Felix Schmeißer\\
Leipzig, den \today
\end{minipage}
\clearpage
\tableofcontents
\chapter{Einleitung}
\subsubsection{Problemstellung}
Es gibt unzählige JavaScript-Frameworks und es kommen ständig neue hinzu. Als Entwickler muss abgewägt werden, welche Lösung für das speziell vorliegende Szenario geeignet ist.

Angular ist ein umfangreiches Frontend-Framework und kann damit nahezu jede Aufgabe in diesem Bereich abdecken. Für sehr viele Problemstellungen gibt es eine Lösung direkt aus dem Framework. 

React als JavaScript-Bibliothek ist deutlich reduzierter. Abseits der elementaren Funktion von React werden externe Bibliotheken oder Frameworks verwendet. Demzufolge muss hier zwischen verschiedenen Möglichkeiten entschieden werden.

\subsubsection{Ziel der Arbeit}
Das Ziel der Arbeit ist, die Technologien umfassend einzuführen und daraus Gemeinsamkeiten und Unterschiede zu erkennen. Auch werden mögliche Vor- und Nachteile hinsichtlich der Architektur angeschnitten. Dazu wird im Rahmen der Arbeit eine Testanwendung mit beiden Frameworks implementiert. Das Ziel ist, die geeigneten Anwendungsszenarien der Technologien daraus abzuleiten.

\subsubsection{Vorgehensweise}
Zunächst erfolgt eine allgemeine Einführung mit der Entstehungsgeschichte, zugrundeliegenden Technologien und dem Aufbau neuer Projekte. Im Anschluss werden die Konzepte von Angular und React beschrieben. Danach wird der Vergleich praktisch durchgeführt. Ziel ist es, eine identische Anwendung sowohl in Angular
als auch in React zu implementieren, um die Unterschiede zu verdeutlichen und Grenzen aufzuzeigen. Abschließend werden die Beobachtungen eingeordnet und Schlüsse über Vor- und Nachteile beider Technologien gezogen.

\subsubsection{Abgrenzung}
Die Arbeit gibt eine fundierte Übersicht über die Grundprinzipien von Angular und React. Eine komplette Abhandlung ist nicht möglich und für die Beantwortung der zentralen Fragestellung auch nicht notwendig.

\subsubsection{Bibliothek und Framework}
Ein Framework gibt einen Rahmen für die Entwicklung vor und bestimmt die Architektur der Anwendung. Einige Frameworks liefern außerdem fertige Komponenten und Klassen, sondern auch Laufzeitumgebungen und Interfaces zum Erstellen neuer Bestandteile, Testsysteme und Auslieferung der Anwendung als Produktionsversion.

Eine Bibliothek ist eine Sammlung von einzelnen Programmteilen wie Klassen oder Komponenten, Routinen oder ganzen Programmen, die für die Entwicklung genutzt werden können und die Arbeit unterstützen. Im Gegensatz zu Frameworks definieren Sie nicht den Rahmen der Anwendung, sondern sind eigenständige Bausteine.

Angular ist ein Webframework, React eine Bibliothek. Im Folgenden werden beide Technologien zur Vereinfachung als Framework zusammengefasst.
\chapter{Allgemeines}

\section{Entstehung}

\subsection{Angular}
\subsubsection{Anfänge als Nebenprojekt}
Die Historie von Angular beginnt 2009 als Nebenprojekt der Entwickler Miško Hevery und Adam Abrons. Ihn störte die aufwändige Entwicklung durch sehr viel Boilerplate-Code, u.a. Entwicklung einer Datenbank, Datenbankzugriff und Sicherheitsvorkehrungen.
Die ursprüngliche Idee bestand also darin, diese Bestandteile zu abstrahieren und ein Framework zu entwickeln, welches von Designern ohne Programmierkenntnisse verwendet werden konnte. Anwendungen sollten unter dieser Prämisse mittels HTML umgesetzt werden können.
\begin{figure}
    \centering
    \includegraphics[scale=0.8]{Grafiken/getangular.png}
    \quelle{Screenshot von: web.archive.org/web/20100227143939/}
    \caption{www.getangular.com}
    \label{fig:getangular}
\end{figure}

\subsubsection{AngularJS}
Hevery arbeitete zu dieser Zeit bei Google mit 2 weiteren Entwicklern an Google Feedback, die Umsetzung wurde mit einem eigenen Java Web Toolkit durchgeführt. In 6 Monaten entstanden 17.000 LOC, der immer schwerer zu warten und testen war. Daraufhin bat Hevery seinen Produktmanager um 2 Wochen Zeit, um das gesamte Projekt mit dem eigenen Framework neu zu entwickeln. Das gelang ihm in 3 Wochen, dabei konnte er den Code um 90\% auf 1.500 LOC reduzieren. Google begann sich für Angular zu interessieren und stellte ein Entwicklerteam. Zunehmend wurden auch interne Projekte mit dem neu benannten AngularJS umgesetzt.\cite{yt_ngConf2014} angularjs: https://angularjs.org/

Im Oktober 2010 wurde AngularJS mit der Versionsnummer 0.9.0 als erste stabile Version auf GitHub veröffentlicht. Bis zur Veröffentlichung der Version 1.0.0 im Juni 2012 war das Framework bereits auf breite Resonanz gestoßen. Das hat verschiedene Gründe, die sich im Ursprung auf die Überlegungen von Hevery und Abrons stützen. Am wichtigsten sind die folgenden Punkte, die AngularJS als erstes clientseitiges Webframework umsetzen konnte:
\begin{itemize}
    \item Dependency Injection sorgt für lose Kopplung und vereinfacht damit Softwaretests
    \item Directives erlauben das Erstellen wiederverwendbarer HTML-Bausteine
    \item Zweiseitiges Data-Binding hält Model und View konsistent
    \item Das Nachladen im Browser wird hinfällig (Single-Page-Application)
\end{itemize}
https://www.ryadel.com/en/angular-angularjs-history-through-years-2009-2019/

Die Version 1.7.0 im Mai 2018 führte als letzter Release zu vorherigen Versionen inkompatible Änderungen ein. \cite{yt_ngConf2014} Bis Juli 2021 wird AngularJS im Long Time Support unterstützt. Es werden lediglich Sicherheitsfehler und durch neue Versionen der gängigen Browser erzwungene Bugs behoben.

https://de.wikipedia.org/wiki/AngularJS

\subsubsection{Angular 2}
2016 veröffentlichte Google den Nachfolger von AngularJS (Angular 2.0) und beschränkte die Bezeichnung auf Angular, da die neue Version als komplett neue Entwicklung nicht kompatibel zu vorhergehenden Releases war. Neu war vor allem der Wechsel zu TypeScript, einem Superset von JavaScript basierend auf ECMAScript6. Des Weiteren verschob sich der Fokus voll auf moderne Browser, um Workarounds zur Unterstützung veralteter Browser zu reduzieren. Viele Konzepte, mit denen AngularJS bereits Vorreiter im Bereich der Webframeworks war, wurden mit Angular weiter verbessert. Templates wurden weiter vereinfacht, u.A. wurden Bindings für Eigenschaften und Events deutlicher syntaktisch getrennt. Weiter wird Kapselung in Modulen und dynamisches Nachladen von Komponenten ermöglicht.

Der letzte Major Release (10.0.0) ist vom Juni 2020. %https://de.wikipedia.org/wiki/Angular

\subsection{React}

React 

TODO: https://blog.risingstack.com/the-history-of-react-js-on-a-timeline/

\section{Verbreitung und Beliebtheit}

Für die Betrachtung der Beliebheit und Verbreitung müssen verschiedene Faktoren bedacht werden. Entwickler stehen zu Beginn neuer Projekte vor dem Problem, die für den Anwendungsfall passenden Technologien auszuwählen. 

\section{Single Page Application}

\section{Verwendete Sprachen}
In den folgenden Abschnitten werden die von Angular und React angewendeten Webtechnologien eingeführt. Grundsätzlich ist das Resultat im Browser das Gleiche: ein HTML-Dokument mit verknüpften CSS-Styles und JavaScript-Code für dynamische Funktionalitäten.

\subsection{HTML-Templates}
HTML gehört mit JavaScript und CSS zu den Grundsäulen von Webseiten und -anwendungen. Die \textbf{H}yper\textbf{t}ext \textbf{M}arkup \textbf{L}anguage gibt den Inhalten eine semantische Struktur und enthält Metainformationen. Grafische Darstellung und dynamische Änderungen werden mit CSS und JavaScript durchgeführt.
%https://de.wikipedia.org/wiki/Datei:Architecture_of_an_Angular_2_application.png

\subsubsection{Angular Template Syntax}

Angular führt eine spezielle Syntax ein, die den aktuellen HTML5-Standard um spezifische Funktionen erweitert. Das \colorbox{lightgray}{<script>} \textit{<script>}-Tag wird mit einer Warnung ignoriert, um Injection-Angriffe zu vermeiden. Sonst sind alle bekannten Tags des Standards erlaubt. Im folgenden Abschnitt werden einige wichtige Syntaxerweiterungen vorgestellt.

Interpolation und Ausdrücke

Mit Interpolation lassen sich Template Expressions (Ausdrücke) in der HTML-Template verwenden:
<h1>Benutzername: {{ userName }}</h1>
Angular wertet die Ausdrücke, welche auch Rechenoperationen oder Funktionsaufrufe beinhalten können aus und ersetzt sie in der finalen Template durch einen String. Die Ausdrücke müssen sich also in Strings umwandeln lassen. Damit sind auch folgende Konstrukte erlaubt:
<span>11 + {{getRandomNumber()}} = {{11 + getRandomNumber()}}.</span>
Die Ausdrücke können neben Interpolationen auch an HTML-Properties gebunden werden:
<img [src]="imageSourceUrl">
Grundsätzlich ist beliebiger JavaScript-Code erlaubt. Nebeneffekte sollen vermieden werden, dadurch fallen Zuweisungen (=, +=, -=, ...), einige Schlüsselwörter (a.A. new, typeof, instanceof), In- und Dekrementierung (++, --) und einige weitere Befehle ab dem ES2015-Standard weg. Auch Bitoperationen sind verboten, neu eingeführt werden der Pipe-Operator (|) zum Verwenden von Pipes und Null-Checks mit ! oder ?. Funktionen können theoretisch weiterhin Nebeneffekte verursachen. Neben der Inter
Die Ausdrücke stehen dabei immer im Kontext der aktuellen Komponente und greifen folglich auf Properties der dazugehörigen TypeScript-Klasse zu. Das Data-Binding ist einseitig von Model zu View. Mehr dazu im folgenden Kapitel.

Das Gegenstück Template Expressions sind Template Statements (Anweisungen). Das Data-Binding ist ebenfalls einseitig von View zu Model. Anweisungen werden ausgeführt, wenn das korrespondierende Event aufgerufen wird, z.B. ein Klick-Event:
<button (click)="login()">Einloggen</button>
Anweisungen erlauben wie Ausdrücke beliebiges JavaScript, allerdings sind Nebeneffekte in diesem Fall der zentrale Punkt, um entsprechende Datenänderungen oder Navigation anzustoßen. Nicht erlaubt ist das Keyword new, In- und Dekrementierung, operative Zuweisungen (+=, -=), Bitoperationen und der Pipe-Operator.
Der Kontext erstreckt sich ebenfalls auf die zugrunde liegende Komponente, erweitert um den Kontext der Template selbst. Anweisungen können folglich auch auf \textit{\$event}-Objekte und Template Variablen zugreifen. Mehr dazu gleich unter Built-in directives.

Banana in a Box
Die letzte Möglichkeit zum Data-Binding ist zweiseitig. Wird also die View verändert, etwa durch Eingabeevents, dann wird das Model geupdated. Anders herum funktioniert das genauso. Ausdrücke werden dazu in zwei Klammern geschrieben:
<input [(ngModel)]="username">
Diese Schreibweise ist nur syntaktischer Zucker und ist mit folgender Schreibweise gleichzusetzen:
<input [ngModel]="username" (ngModelChange)="username = \$event">
<input [value]="username" (input)="username = \$event.target.value">
Damit werden die oben beschriebenen Bindings verwendet.
https://angular.io/guide/template-syntax
https://blog.thoughtram.io/angular/2016/10/13/two-way-data-binding-in-angular-2.html

\subsubsection{JSX (React)}



\subsection{JavaScript}
JavaScript (JS) ist die Programmiersprache des Internets. Alle gängigen Browser besitzen eine Engine zum Kompilieren von JavaScript-Code, Google Chrome setzt beispielsweise auf das eigene Open-Source-Projekt V8. JS wurde ursprünglich im Jahr 1995 von Brendan Eich entwickelt. 1997 wurde AngularJS als ECMAScript (ES) normiert. In den folgenden Jahren gab 2 weitere Versionen des Standards, 2009 kamen mit der 5. Version größere Änderungen. Die nächste Version war ES6 im Jahr 2015, ab hier gab es jährlich neue Releases.

%https://www.w3schools.com/js/js_versions.asp}

\subsubsection{Type Checking}

JavaScript ist eine dynamisch typisierte Sprache. Typen werden zu Laufzeit bestimmt und müssen nicht angegeben werden:

\begin{verbatim}
  var sampleNumber = 123;

  splitStringCharacter(sampleNumber); // -> Laufzeit-Fehler
\end{verbatim}

Damit geht die Programmierung nur vermeintlich schneller, denn Typfehler werden erst zur Laufzeit erkannt, auch wenn der Code vorher kompiliert werden konnte.

Statisch typisierte Sprachen fordern die Angabe von Typen und kompilieren bei Typ-Fehlern nicht:

\begin{verbatim}
  string sampleString = 123; // -> Compiler-Fehler

  number sampleNumber = 123;
  splitStringCharacter(sampleNumber) // -> Compiler-Fehler
\end{verbatim}

Es gibt mehrere Möglichkeiten, JavaScript typsicher zu machen und Laufzeit-Fehler zu reduzieren. 

\subsubsection{TypeScript}
Angular forciert die Nutzung des JavaScript-Supersets TypeScript (TS), ebenfalls eine Implementierung des ECMAScript-Standards und entwickelt von Microsoft. Es erweitert JavaScript mit zusätzlichen Features. JavaScript-Code ist valides TypeScript, nicht umgekehrt. TS wird mit seinem Compiler in gültigen JavaScript-Quelltext in einer gewünschten Zielversion ab ES3 kompiliert.

Die neuen Features von TypeScript kennen Entwickler aus der objektorientierten Programmierung. TypeScript setzt, wie der Name schon andeutet, auf strengere Typisierung.
Damit wird der Entwickler unterstützt und die Code-Qualität erhöht. Zwar gehen die Typings nach der Übersetzung verloren, allerdings werden Laufzeitfehler durch die vorgeschobene Kontrolle bereits bei der Entwicklung mit TypeScript reduziert.

https://blog.doubleslash.de/was-ist-eigentlich-typescript/

https://de.wikipedia.org/wiki/TypeScript

\subsubsection{Flow}

React legt sich nicht auf einen Type Checker fest. In der Dokumentation wird sowohl die Installation von TypeScript, als auch Flow beschrieben. Flow ist keine eigenständige Sprache mit Compiler wie TypeScript, sondern ermöglicht Type Checks durch Annotationen. Mit dem Befehl \textit{flow} werden markierte JavaScript-Dateien überprüft. Flow wird wie React von Facebook entwickelt.
https://flow.org/en/docs/getting-started/
https://reactjs.org/docs/static-type-checking.html

\subsection{CSS}

Cascading Style Sheets, kurz CSS, gehört neben HTML und JavaScript zu den 3 Hauptsprachen des Open Web. Mit CSS werden die mit HTML semantisch strukturierten Elemente formatiert, das Styling bleibt dadurch unabhängig von den Inhalten. CSS wird vom W3C standardisiert, Version 3 seit 2000 fortlaufend weiterentwickelt. Neue Versionen gibt es nur noch für einzelne CSS-Module, nicht für den gesamten Standard.
TODO: https://developer.mozilla.org/de/docs/Web/CSS
https://www.xanthir.com/b4Ko0

Die Arbeitsweise mit CSS selbst ändert sich für React und Angular im Gegesatz zu HTML und JavaScript nicht. Entwickler können frei unter verschiedene Präpozessoren oder anderen Lösungen entscheiden. Allerdings ändert sich die Verwendung innerhalb der HTML-Templates.

In React werden Klassen mit dem className-Attribut zugewiesen: 
\begin{verbatim}
render() {
  return <span className="headline-big highlighted">Headline</span>
}
\end{verbatim}

Durch JSX ergibt sich die Möglichkeit, Styles dynamisch hinzuzufügen: 

\begin{verbatim}
render() {
  let className = 'button-save';
  if (this.props.isForwardable) {
    className += 'button-highlighted';
  }
  return <button className={className}>Activate</button>
}
\end{verbatim}

Das style-Attribut ist erlaubt, allerdings sollten solche Inline-Styles vermieden und eigene Stylesheets verwendet werden (TODO % https://reactjs.org/docs/dom-elements.html#style).

TODO https://reactjs.org/docs/faq-styling.html

Angular erlaubt weiterhin die Nutzung von class für HTML-Elemente. Mit Class-Binding können Strings, String-Arrays oder Objekte angegeben werden. Das Objekt ist eine Liste aus Key-Value Paaren mit einem String aus CSS-Klassen und einem Boolean, der angibt, ob die Klassen aktiv sind oder nicht.

\begin{verbatim}
  <some-element class="first second">...</some-element>

  <some-element [class]="'first second'">...</some-element>

  <some-element [class]="['first', 'second']">...</some-element>

  <some-element [class]="{'first': true, 'second third': false}">...</some-element>
\end{verbatim}

Class-Binding kann sich auch auf einzelne Klassen beziehen: 

\begin{verbatim}
  [class.menu-active]="isMenuActive"
\end{verbatim}

Inline-Styles sind auch in Angular erlaubt und mit dem Style-Binding gibt es erweiterte Möglichkeiten. Man kann direkt auf CSS-Attribute zugreifen: 

\begin{verbatim}
  [style.width]="100px" 
  [style.width.px]="100" 
  [style]="styleExpr"

  styleExpr -> "width: 100px; height: 100px" 
  oder -> {width: '100px', height: '100px'}
  oder -> ['width', '100px']
\end{verbatim}

Allerdings ist auch hier von der Nutzung von Inline-CSS abzuraten. Externe Stylesheets sind leichter zu pflegen, eine Vermischung mit Inline-Styles erschwert die Fehlersuche und bricht die Trennung von View-Content und Design.

% TODO https://angular.io/guide/attribute-binding#class-binding
https://angular.io/api/common/NgClass

\subsubsection{Sass}
Sass vereinfacht CSS und fügt Funktionalitäten hinzu, die die Les- und Wartbarkeit der Style Sheets verbessern, beispielsweise mit Mixins: CSS-Blöcke können als Variablen definiert und an anderen Stellen verwendet werden. Man bezeichnet Sass als Präpozessor, weil ein Compiler die erweiterte Syntax zu nativem CSS verarbeitet. Neben Sass gibt es noch weitere Präpozessoren wie LESS, Stylus oder PostCSS. 

%TODO  https://developer.mozilla.org/de/docs/Glossary/CSS_Praeprozessor


\section{Projekterstellung}
\subsection{Vorgehensweise}
\subsubsection{Angular CLI}
Die Erstellung eines Angular Projektes erfolgt mit dem Angular CLI (Command Line Interface), welches mit dem Node Package Manager (NPM) installiert wird: \textit{npm install -g @angular/cli}. Das CLI verfügt über einige Befehle, die beispielsweise externe Bibliotheken hinzufügen (\textit{ng add <Bibliothek>}), neue Bestandteile (z.B. Components) anlegen (\textit{ng generate <Schema>}) oder Tests starten (\textit{ng test}). Mit \textit{ng new <Projektname>} wird ein neues Projekt in einem entsprechend benannten Ordner angelegt. Das Projekt ist hier schon startbereit, mit \textit{ng serve --open} kann man die Anwendung im Development Modus starten, durch das Flag wird automatisch der Browser geöffnet (http://localhost:4200/.). Änderungen im Projekt werden erkannt und die Anwendung sofort aktualisiert.
https://angular.io/guide/setup-local

\subsubsection{React Toolchains}
React benötigt als grundsätzlich reine JavaScript-Bibliothek kein umfangreiches Setup. Die einfachste Möglichkeit der Einbindung besteht darin, React innerhalb des \textit{<script>}-Tags einzubinden. Damit lässt sich React zu bestehenden Projekten hinzufügen, um etwa eine schrittweise Migration zu ermöglichen. Für neue Projekte die von Beginn an mit React entwickelt werden sollen, bietet sich eine Toolchain für das entsprechende Anwendungsszenario an. Damit wird die Umgebung für die React-Entwicklung optimiert: Häufig wird bereits eine Grundstruktur und Scripte zum Starten der Anwendung mit erweiterten Debugoptionen generiert. Die einfachste Toolchain stammt ebenfalls von Facebook und heißt \textit{create-react-app}. Mit dem Befehl \textit{npx create-react-app <Projektname>} wie beim Angular CLI ein Projektordner generiert. Durch \textit{npm start} wird die Applikation gehostet (http://localhost:3000/)
https://github.com/facebook/create-react-app
https://reactjs.org/docs/create-a-new-react-app.html\#create-react-app

\subsection{Aufbau}

\chapter{Technischer Vergleich}

\section{Components}

\subsection{Komponentenmodell}

\subsection{Angular}

\subsection{React}

\section{Weitere Features}
\subsection{Angular}
\subsection{React}

\section{Lifecycle}
\subsection{Angular}
\subsection{React}

\section{DOM}
\subsubsection{Regular DOM}
\subsubsection{Virtual DOM}
\subsubsection{Vergleich}

\section{Integrierte Design Pattern}
\subsection{Dependency Injection}
\subsection{Services}
\subsection{Decorators}
\subsection{Pipes}
\subsection{State Management}

\section{Projektarchitekturen}
\chapter{Implementation}\label{implementation}
Das folgende Kapitel dokumentiert die wichtigsten Bestandteile der Implementation einer Testanwendung mit den betrachteten Frameworks. Dabei werden grundsätzliche Konzepte durch einfache Anwendungsfälle abgedeckt. Das Ziel sind nach Möglichkeit identische Webapplikationen, um die Vorgehensweise und Probleme vergleichen zu können. Sie hat nicht den Anspruch, alle erdenklichen Use-Cases abzudecken. Auf Drittbibliotheken- und Frameworks wird weitestgehend verzichtet. 
Der vollständige Quelltext ist auf GitHub zu finden\footnote{siehe: https://github.com/fschmeis/bthesis\_react\_angular}, im diesem Kapitel werden nur Ausschnitte abgebildet.

\section{Anforderungsbeschreibung}
Die Anwendung soll Rezepte und Zutaten verwalten und erfüllt genauer die folgenden Anforderungen:
\begin{enumerate}
  \item Rezepte können mit Namen und Anweisungen, benötigten Zutaten und Mengen eingetragen werden
  \item Es können neue Zutaten hinzugefügt werden
  \item Eine Übersicht zeigt die gespeicherten Zutaten und Mengen an
  \item Eine Übersicht zeigt Rezepte und gibt an, ob die geforderten Zutaten in ausreichender Menge vorhanden sind
\end{enumerate}

Die Applikation verfügt über 3 mit Tabs navigierbare Ansichten für die Anforderungen 1, 3 und 4. Die Eingabe der Zutaten erfolgt über einen Dialog. Über den Tabs liegt eine einfache Leiste mit dem Titel der Anwendung.

\subsubsection{Rezepte hinzufügen}
Der Nutzer kann im linken Bereich der Ansicht über ein Formular den Titel und die Anweisen des Rezeptes eingeben. Auf der rechten Seite ist ein Button, der beim Klick den Zutaten-Dialog öffnet. Die bereits hinzugefügten Zutaten werden darunter in einer Liste angezeigt.

Unter diesem Formular sind zwei Buttons angeordnet: Ein Button zum Speichern des Rezeptes, der andere zum Zurücksetzen des Formulars.

\subsubsection{Zutaten}
Der Nutzer sieht eine Tabelle der angelegten Zutaten mit Einheit und verfügbarer Menge. Über einen Button darüber öffnet sich der Zutaten-Dialog, dort können neue Zutaten hinzugefügt werden. Die Tabelle lässt sich sortieren und filtern.

\subsubsection{Kochbuch}
Hier werden die angelegten Rezepte seitenweise angezeigt. Links stehen Titel und Anweisungen, rechts eine Liste mit benötigten und vorhandenen Zutaten. Ist alles nötige vorhanden, dann werden mit Klick auf den Button "Kochen" die Mengen der verbrauchten Zutaten entsprechend reduziert.

Unter dem Rezept kann über Pfeile geblättert werden. Darüber ist einer Filterleiste, um nach bestimmten Rezepten zu suchen.

\subsubsection{Zutaten-Dialog}
Ein einfacher Dialog, der Zutatenname, Einheit und Menge abfragt. Mit dem Button \glqq Hinzufügen\grqq werden die Eingaben bestätigt, ein weiterer Button ermöglicht den Abbruch der Aktion.

\subsection{Umsetzung}
Die Umsetzung soll möglichst ohne Drittlösungen auskommen, um die möglichen Grenzen der Standard-Versionen aufzuzeigen. Aus den Material Design Frameworks Angular Material\footnote{siehe: https://material.angular.io} und Material UI (React) \footnote{siehe: https://material-ui.com} werden sowohl einfache Komponenten als auch Schriftarten und Themes übernommen. Letzteres besitzt deutlich mehr Komponenten, daher wird mit Angular begonnen. Für das Styling wird Sass verwendet.

\section{Angular}
\subsection{Dateistruktur}
Das Projekt wird in verschiedene Teilbereiche strukturiert. Die Hauptkomponente der Anwendung (\textit{app.component.html|ts|scss}) befindet sich automatisch direkt im \textit{app}-Ordner und beinhaltet den Einstiegspunkt der Anwendung. Unter \textit{app/components/} finden sich die 3 Hauptkomponenten der Anwendung. In einem weiteren Ordner (\textit{app/shared/components}) befindet sich die Dialog-Komponente. Der Dialog wird an zwei Stellen benötigt und es ist zur Übersicht sinnvoll, diesen Umstand schon in der Dateistruktur ersichtlich zu machen. Im \textit{shared}-Ordner befinden sich außerdem die Models und Services. 

Die Implementation in Angular mit den entsprechenden Vorüberlegungen unkompliziert. Das Angular CLI erzeugt automatisch neue Komponenten oder Services und fügt sie dem App-Modul hinzu. 

\subsection{Models}
Die Models bilden das grundlegende Datenmodell der Anwendung ab (Listing \ref{lst:listing18}). Unter \textit{Unit} werden die verfügbaren Einheiten (string) deklariert, die der Nutzer für Zutaten wählen kann. Eine Zutat \textit{Ingredient} definiert sich aus Name, Anzahl und Einheit, ein Rezept (\textit{Recipe}) aus Name, Anweisungen und Zutaten.

\begin{listing}
\caption{Datenmodell}
\label{lst:listing18}
\begin{minted}{ts}
// unit.enum.ts
export enum Unit {
  tl = 'TL',
  el = 'EL',
  stk = 'Stück',
  bnd = 'Bund'
  /* ... */
}
// ingredient.model.ts
export interface Ingredient {
  name: string;
  amount: number;
  unit: Unit;
}
// recipe.model.ts
export interface Recipe {
  name: string;
  instructions: string;
  ingredients?: Ingredient[];
}
\end{minted}
\end{listing}

\subsection{Services}\label{sssec:impl_services}
Die Anwendung besitzt zwei Services als Schnittstelle zum Backend, die mit dem HTTP-Client auf die Backend-API zugreifen. Die Services werden mit der beschriebenen Dependency Injection über den Konstruktor übergeben und können dann in den Komponenten genutzt werden. Im Listing \ref{lst:listing19} in der \inlinecode{OnInit}-Methode, die bei der Initialisierung der Komponente aufgerufen wird.

\begin{listing}
\caption{Verwendung der Services}
\label{lst:listing19}
\begin{minted}{ts}
// cookbook.component.ts
export class CookbookComponent implements OnInit {
  /* ... */
  constructor(
    private recipeService: RecipeService,
    private ingredientService: IngredientService,
    /* ... */
  ) {}

  public ngOnInit(): void {
    this.fetchData();
  }

  private fetchData(): void {
    this.recipeService.getRecipes().subscribe((recipes) => {
      this.recipes = recipes;
      this.currentRecipe = this.recipes[0];
    });

    this.ingredientService.getIngredients().subscribe((ingredients) => {
      this.storedIngredients = ingredients;
    });
  }
  /* ... */
}
\end{minted}
\end{listing}

\subsection{Komponenten}
Die Anwendung besteht aus 4 Komponenten. Die \textit{app}-Komponente ist die Hauptkomponente und Startpunkt der Anwendung. Sie enthält die Toolbar mit dem Anwendungstitel und eine Tab-Leiste, über welche die anderen Teilkomponenten erreicht werden können (Abbildung \ref{fig:kochbuch_angular}).

\subsubsection{Kochbuch}
\begin{figure}
  \centering
  \includegraphics[scale=0.6]{Grafiken/03_Implementation/Kochbuch_Angular_edited.png}
  \caption{Kochbuch}
  \label{fig:kochbuch_angular}
\end{figure}

Die Komponente enthält neben der Service-Abfrage aus Listing \ref{lst:listing19} die ViewModel-Logik zum Vor- und Zurückblättern der Rezepte, zur Berechnung der fehlenden Zutatenmenge und Entnahme der Zutaten, wenn das Rezept gekocht wird. Die Template (Auszug im Listing \ref{lst:listing20}) definiert lediglich die Darstellung des ViewModels. Die mit \inlinecode{mat} beginnenden Tags referenzieren Komponenten der Material-Bibliothek. 

\begin{listing}
\caption{Template der Zutaten-Checkliste}
\label{lst:listing20}
\begin{minted}{html+ng2}
<div *ngIf="insufficientIngredients(); else cook">
  <span>Dir fehlen:</span>

  <mat-list role="list">
    <mat-list-item role="listitem"
      *ngFor="let ing of currentRecipe.ingredients; last as isLast">

      <ng-container *ngIf="getMissingAmount(ing) > 0; else tick">
        {{getMissingAmount(ing)}} {{ing.unit}} {{ing.name}}
      </ng-container>

      <ng-template #tick>
        <mat-icon id="tick">done</mat-icon>
      </ng-template>

      <mat-divider *ngIf="!isLast"></mat-divider>

    </mat-list-item>
  </mat-list>
</div>

<ng-template #cook>
  <div id="cook">
    <button mat-raised-button color="primary"
      (click)="cookRecipe()">Kochen
    </button>
  </div>
</ng-template>
\end{minted}
\end{listing}

In Zeile 6 wird mit \inlinecode{ngFor} über alle Zutaten des aktuellen Rezeptes iteriert, für jede dieser Zutaten wird ein neues \inlinecode{mat-list-item} erzeugt, welches entweder Text (fehlende Menge, Einheit und Zutat) oder ein Häkchen-Icon enthält. Bestimmt wird das durch die Abfrage in Zeile 8, sind genug Zutaten vorhanden, dann wird die Template-Variable \inlinecode{tick} wahr und das Icon gerendert. \inlinecode{ng-template} und \inlinecode{ng-container} sind strukturelle Direktiven, die selbst nicht gerendert werden.

\subsubsection{Zutaten}
\begin{figure}
  \centering
  \includegraphics[scale=0.5]{Grafiken/03_Implementation/Zutaten_Angular.png}
  \caption{Zutaten-Übersicht}
  \label{fig:zutaten_angular}
\end{figure}

Die Übersicht der vorhandenen Zutaten besteht aus einer Tabellenkomponente der Material-Bibliothek und einem Button, der den Zutaten-Dialog öffnet. Die Material-Tabelle verfügt über die Möglichkeit, Spalten zu sortieren und durch Einträge zu schalten. Eingaben in der Suchleiste filtern die Datenquelle (Listing \ref{lst:listing21}).

\begin{listing}
\caption{Setzen und Filtern der Datenquelle}
\label{lst:listing21}
\begin{minted}{ts}
export class IngredientsComponent {
  dataSource!: MatTableDataSource<Ingredient>;
  /* ... */
  private fetchIngredients(): void {
    this.ingredientService.getIngredients().subscribe((ingredients) => {
      this.dataSource = new MatTableDataSource(ingredients);
    });
  }
  /* ... */
  public applyFilter(event: Event): void {
    const filterValue = (event.target as HTMLInputElement).value;
    this.dataSource.filter = filterValue.trim().toLowerCase();

    if (this.dataSource.paginator) {
      this.dataSource.paginator.firstPage();
    }
  }
}
\end{minted}
\end{listing}

\subsubsection{Rezepte hinzufügen}
\begin{figure}
  \centering
  \includegraphics[scale=0.5]{Grafiken/03_Implementation/Hinzufuegen_Angular.png}
  \caption{Rezepte hinzufügen}
  \label{fig:hinzufuegen_angular}
\end{figure}

Die Komponente definiert Methoden, die das Formular zurückzusetzen, Zutaten hinzuzufügen oder entfernen und das Rezept bei entsprechender Nutzereingabe zu speichern.

\subsubsection{Zutaten-Dialog}
\begin{figure}
  \centering
  \includegraphics[scale=0.5]{Grafiken/03_Implementation/Dialog_Angular.png}
  \caption{Zutaten-Dialog}
  \label{fig:dialog_angular}
\end{figure}

Der Dialog wird aus anderen Komponenten heraus aufgerufen. Mit Angular Formularen lassen sich die eingegeben Werte validieren. Das Ergebnis des Dialogs wird im Anschluss wieder an die aufrufende Komponente zurückgegeben (Listing \ref{lst:listing22}).

\begin{listing}
\caption{Dialogaufruf und Rückgabe der Daten}
\label{lst:listing22}
\begin{minted}{ts}
// add-recipe.component.ts
public onClickAddIngredient(): void {
  this.dialog
    .open(AddIngredientDialogComponent)
    .afterClosed()
    .subscribe((newIngredient) => {
      this.updateIngredients(newIngredient as Ingredient);
    });
}

// add-ingredient-dialog.component.ts
// Nutzereingabe wird zurückgegeben
add(): void {
  const newIngredient = {
    ...this.ingredientFormGroup.value,
    id: 0
  } as Ingredient;
  this.dialogRef.close(newIngredient);
}
\end{minted}
\end{listing}

\section{React}
Für die Implementation mit React müssen durch die Einfachheit der Bibliothek einige Vorüberlegungen getroffen werden. Aufgrund der geringen Komplexität der Testanwendung wird auf weitere Lösungen wie Redux verzichtet. Die Services aus der Angular-Implementation werden übernommen. Die Root-Komponente reicht den für die Views relevanten Teil des States (z.B. Zutaten für die Übersicht) zusammen mit einer Callback-Funktion an die Komponenten weiter, über welche eine Aktualisierung der Daten erreicht wird. Die Komponenten werden als Funktionen mit Hooks geschrieben.

Für die React-Implementation wird ebenfalls TypeScript verwendet. Die Vorteile einer statischen Typisierung wurden bereits erläutert, die Entscheidung fällt aus Zeitgründen auf TypeScript. Die Dateien haben statt \inlinecode{.jsx} die Endung \inlinecode{.tsx}.

Auf Abbildungen wird verzichtet, da die Anwendung bis auf die Zutatenübersicht identisch aussieht. Die Tabellenkomponente aus der React Material UI erfordert im Gegensatz zu Angular größeren Aufwand, aus diesem Grund wird hier zusätzlich \inlinecode{material-table}\footnote{siehe: https://material-table.com} als fertige Komponente verwendet. Die Funktionalität ist gleich, das Aussehen etwas verändert. 

\subsection{Dateistruktur}
Die Einstiegskomponente \inlinecode{App.tsx} befindet sich der Anwendung  Unter \textit{app/components/} finden sich die 3 Hauptkomponenten der Anwendung. In einem weiteren Ordner (\textit{app/shared/components}) befindet sich die Dialog-Komponente. Der Dialog wird an zwei Stellen benötigt und es ist zur Übersicht sinnvoll, diesen Umstand schon in der Projektstruktur ersichtlich zu machen. Im \textit{shared}-Ordner befinden sich außerdem die Models und Services. 

Die Implementation in Angular mit den entsprechenden Vorüberlegungen unkompliziert. Das Angular CLI erzeugt automatisch neue Komponenten oder Services und fügt sie dem App-Modul hinzu. 

\subsection{Übernommene Teile}

Die Models werden übernommen. Services ebenfalls, allerdings wird eine einzelne Service-Instanz exportiert: \mintinline{ts}{export const ingredientService = new IngredientService();}. Diese Instanz kann dann importiert werden. Die Styles werden ebenfalls übernommen, allerdings erfordern die Unterschiede der Material-UI Bibliotheken entsprechende Anpassungen der Selektoren. 

\subsection{Komponenten}

\subsubsection{MainTabs}
Die \inlinecode{MainTabs}-Komponente ist die Elternkomponente für alle von Daten abhängigen Komponenten, sie reicht diese zusammen mit Callbacks an die Kindkomponenten. Die Callbacks sind einfache Funktionen, welche Rezepte respektive Zutaten entgegen nehmen und an die entsprechenden Services zur Speicherung übergeben (Listing \ref{lst:listing23}).

\begin{listing}
\caption{MainTabs und Kindkomponenten}
\label{lst:listing23}
\begin{minted}{jsx}
<!-- MainTabs.tsx -->
<TabPanel value="1">
    <Cookbook isLoading={areRecipesLoading} 
              recipes={recipes} 
              storedIngredients={storedIngredients} />
</TabPanel>
<TabPanel value="2">
    <Ingredients storedIngredients={storedIngredients} 
                 handleIngredient={handleIngredient} />
</TabPanel>
<TabPanel value="3">
    <AddRecipe handleRecipe={handleRecipe} />
</TabPanel>
<!-- ... -->
\end{minted}
\end{listing}

\subsubsection{Kochbuch}
Diese Komponente beinhaltet die Navigation zwischen den Rezepten und eine weitere Komponente, welche die Rezepte abbildet. Diese \inlinecode{RecipePage}-Komponente bringt den Rezeptnamen und die Anweisungen zur Darstellung und enthält ebenfalls eine weitere Komponente. Die \inlinecode{IngredientChecklist} rendert die Zutatenlisten (siehe \ref{fig:kochbuch_angular}).

\subsubsection{Zutaten}
Hier kommt das \inlinecode{material-table}-Paket zum Einsatz. Die externe Komponente ist ein gutes Beispiel für die generelle Problematik, die durch Drittlösungen entstehen kann. In der Konsole erscheint ein Error, wenn die Tabelle gerendert wird, auch wenn er keine sichtbarem Folgen hat: \glqq React does not recognize the `scrollWidth` prop on a DOM element\grqq . Das Paket nutzt ein Prop, das in der aktuellen Material-Version nicht mehr existiert.

Über Props kann die Tabelle konfiguriert werden, um unter anderem die Daten zur Verfügung zu stellen, die Spalten zu definieren oder deutsche Übersetzungen der Label anzugeben. Der Zutaten-Dialog wird ebenfalls im JSX deklariert, aber erst gerendert, wenn das \inlinecode{open}-Attribut einen wahren Wert übergeben bekommt (Listing \ref{lst:listing24}).

\begin{listing}
\caption{Zutaten-Übersicht mit material-table}
\label{lst:listing24}
\begin{minted}{jsx}
// Daten und Callback aus der Elternkomponente MainTabs
interface IngredientsProps {
    storedIngredients: Ingredient[];
    handleIngredient: (Ingredient: Ingredient) => void;
}

export default function Ingredients(props: IngredientsProps) {
    const [dialogOpen, setDialogOpen] = useState(false);

    return (
        <Paper className="tab-paper">
            <MaterialTable
                data={props.storedIngredients}
                actions={[
                    {
                        icon: "add_shopping_cart",
                        tooltip: "Zutat hinzufügen",
                        isFreeAction: true,
                        onClick: () => setDialogOpen(true),
                    },
                ]}
            />
            {/* weitere Optionen ... */}

            <IngredientDialog
                open={dialogOpen}
                setOpen={setDialogOpen}
                onConfirm={props.handleIngredient}></IngredientDialog>
        </Paper>
    );
}
\end{minted}
\end{listing}

\subsubsection{Rezepte hinzufügen}

Die Komponente ist mit 142 Zeilen deutlich größer als die anderen, aber nicht zu lang. Ähnlich zum Kochbuch hätten einzelne Teile in weitere Komponenten ausgelagert werden können. Man muss hier entsprechend abwägen, das Kochbuch ist zwar übersichtlicher, allerdings muss zwischen mehreren Dateien gesprungen werden.

\subsubsection{Zutaten-Dialog}

Die Einbindung des Zutaten-Dialogs wurde bereits in Listing \ref{lst:listing24} gezeigt. Die Komponente selbst setzt ebenfalls stark auf die Material-Komponenten (Listing \ref{lst:listing25}).

\begin{listing}
\caption{Zutaten-Dialog}
\label{lst:listing25}
\begin{minted}{jsx}
// IngredientDialog.tsx
/* ... */
const handleChange = (name: string) =>
    (event: ChangeEvent<{ value: unknown }>) => {
        setValues({ ...values, [name]: event.target.value });
};

const nameError = values.name === "";

return (
  {/* ... */}
  {/* Eingabefeld für den Zutatennamen */}
  <TextField
      required
      value={values.name}
      label="Name"
      onChange={handleChange("name")}
      error={nameError}
      helperText={nameError ? "Bitte gib eine Zutat an." : ""}
  />
  {/* ... */}
  {/* Bestätigung der Eingabe */}
  <Button
      onClick={() => {
          setOpen(false);
          onConfirm({
              name: values.name,
              amount: values.amount,
              unit: values.unit as Unit,
          });
      }}
      color="primary"
      variant="contained">
      Hinzufügen
  </Button>
\end{minted}
\end{listing}
\chapter{Auswertung}
\section{Vergleich}
\subsubsection{Komponenten}
Komponenten in Angular bestehen aus Klassen mit einem dazugehörigen Template. Lifecycle-Methoden kann der Entwickler für bestimmte Aufgaben (z.B. Backend-Anfragen) verwenden.
Die Template Syntax erweitert HTML, um Komponenten und Direktiven verwenden zu können, auf Eigenschaften der zugrundeliegenden Komponente zuzugreifen und bei Änderung in der View aktualisieren (bidirektionale Bindung). Mitgelieferte Direktiven wie \inlinecode{ngFor} oder \inlinecode{ngIf} ermöglichen die dynamische Generierung von HTML, Pipes verarbeiten Daten auf Ebene der Templates. Mit In- und Outputs können Komponenten Daten in der Hierarchie nach unten oder oben weitergeben.

Komponenten in React bestehen aus Klassen, die eine render-Funktion bereitstellen, oder Funktionen, die JSX zurückliefern. Auf den Lifecycle kann für Klassen ebenfalls mit Methoden, für Funktionen mit React Hooks zugegriffen werden. Die Templates werden als JSX direkt in JavaScript definiert, Direktiven und Pipes gibt es nicht. Eigenschaften/Variablen der Klasse/Funktion können direkt verwendet werden. Mit Props werden Daten oder Callback-Funktionen an Kindelemente weiter gereicht, Komponenten werden bei Änderungen neu gerendert (unidirektionale Bindung).

Angular setzt auf Services, um von Komponenten unabhängige Logik abzukapseln. Sie werden per Dependency Injection bereitgestellt. React hat dafür keine Entsprechungen und überlässt das dem Entwickler.

\subsubsection{Change Detection und Performance}
Für die Change Detection verwenden beide Frameworks unterschiedliche Algorithmen, die in der Vergangenheit viele Optimierungen erhalten haben. Bezogen auf die Geschwindigkeit kann hier keine gesicherte Aussage getroffen werden, da dazu keine Benchmarks gefunden werden konnten. Häufig wird nur die initiale Ladezeit in Betracht gezogen. Der Unterschied ist dabei nicht signifikant und davon abhängig, welche zusätzlichen Frameworks mit React verwendet werden\cite{Benchmark}. In der Kochbuch-Testapplikation waren die Übergänge in der Angular-Variante spürbar flüssiger. Daraus sollte allerdings kein genereller Schluss gezogen werden, aufgrund der geringeren Erfahrung mit React könnten eventuelle Fehler die Performance beeinträchtigen.

\subsubsection{Projektarchitektur}
Angular orientiert sich an klassischen MV*-Mustern und erweitert sie mit Services. Projekte werden mit Modulen strukturiert. Komplexes State-Management kann mit Services gelöst werden, alternativ werden Frameworks wie Redux verwendet. Für viele Aufgaben gibt es eine Lösung direkt aus dem Framework: Internationalisierung, komplexe Formulare, eine integrierter HTTP Client oder Animation, nur in seltenen Fällen sind weitere Frameworks notwendig.

React hat keine Module oder Services, JSX ist per Definition gegensätzlich zu MV*Mustern. Für komplexes State-Management muss eine eigene Lösung implementiert oder ebenfalls auf externe Bibliotheken/Frameworks wie Redux zugegriffen werden. Diese Abhängigkeit von externen Lösungen gilt für die meisten Aufgaben, beispielsweise Internationalisierung (react-intl), HTTP-Clients (axios) oder Animationen (react-animations).

\subsection{Lernkurve}
Die benötigte Einarbeitungszeit ist subjektiv und abhängig von den Fähigkeiten und der Erfahrung des Entwicklers. Trotzdem lässt sich die These aufstellen, dass React an sich eine flachere Lernkurve besitzt. Der Grund liegt im deutlich kleineren Funktionsumfang von React. Es gibt keine Dependency Injection, Services, Module oder Direktiven, TypeScript ist keine Pflicht. Selbst für kleinere Angular-Projekte sollte grundsätzliches Verständnis vorliegen. Wird React mit zusätzlichen Bibliotheken als Einheit gegenüber Angular betrachtet, dann nähern sich die Lernkurven an.

\subsection{Anwendungsbereiche}
Angular ist für große Enterprise-Anwendungen geeignet, da es out-of-the-box beliebig skalierbar ist. Bemerkbar macht sich dieser Umstand in der Größe, unverpackt verbraucht Angular 566KB Speicher, React lediglich 97,5KB\footnote{siehe: https://gist.github.com/Restuta/cda69e50a853aa64912d}. Bei kleineren Projekten stellt sich dann die Frage, ob ein Framework dieser Größe notwendig ist. Die notwendige Boilerplate für Dekoratoren und Module macht einen größeren Anteil aus.

React bringt selbst nur einen Bruchteil an Lösungen mit sich, das wirkt sich entsprechend auf die Größe aus. Der Vorteil ist, dass sich React auch für kleine Projekte anbietet. Mit rein funktionalen Komponenten ohne State lassen sich in sehr kurzer Zeit dynamische Websites erstellen, durch JSX muss nicht zwischen HTML und JavaScript hin und her gesprungen werden. Unter Zuhilfenahme zusätzlichen Frameworks sind React-Projekte trotzdem beliebig skalierbar. Der Nachteil hierbei ist allerdings, dass das Projekt dadurch zusätzlichen Abhängigkeiten unterliegt, da die Pakete ebenfalls up-to-date gehalten werden müssen. Der Verwaltungsaufwand ist in diesem Bereich also höher. Bei Angular-Projekten muss in der Regel nur das Framework selbst aktualisiert werden.

Die Entscheidung für oder gegen Angular respektive React ist am Ende immer abhängig von der zu lösenden Aufgabe und den gegebenen Voraussetzungen. React ist grundsätzlich in jedem Szenario verwendbar, aber je komplexer die Aufgabe ist, desto durchdachter muss die gewählte Architektur im Zusammenspiel verschiedener Frameworks sein. Angular ist für kleinere Websites mit statischen oder nur begrenzt dynamischen Inhalten ungeeignet, weil ein Großteil der gebotenen Lösungen nicht benötigt wird. Für Unternehmen ist die Entscheidung auch vom Personal abhängig, sie müssen sich der Fähigkeiten und Erfahrungen der beteiligten Entwickler bewusst sein. Das kann im Zweifel schwerer gewichtet sein, als die Eignung der Technologie.
\chapter{Fazit}
Angular und React sind zwei Technologien, mit deren Hilfe dynamische Webanwendungen entwickelt werden können. Die einzelnen Bestandteile des Frontends werden in wiederverwendbare Komponenten aufgeteilt. Beide erweitern klassisches HTML mit einem Templating-Ansatz, um eine stärkere Bindung mit JavaScript zu erreichen. Dadurch ist direkte DOM-Manipulation mit jQuery oder JavaScript selbst nicht mehr notwendig.

React bietet als für das Rendern von Komponenten optimierte Bibliothek nicht viel mehr als die genannten Punkte. Für komplexere Anwendung ist man auf Drittlösungen angewiesen. Das erhöht den Wartungsaufwand. Mit React können aber in kurzer Zeit dynamische Websites oder Webanwendungen gebaut werden.

Angular macht dem Entwickler mehr Vorgaben, es finden sich für gängige Problemstellungen immer mitgelieferte Lösungen. Diese sind allerdings immer enthalten, unabhängig davon, ob sie auch benötigt werden. Angular erfordert deutlich mehr Boilerplate, allerdings wird im Gegensatz zu React eine gute Basis auch für komplexere Anwendungen geboten. Das Angular CLI unterstützt die Generierung neuer Komponenten.

Diese Arbeit hat die Kernelemente der Frameworks durch eine tiefere Vorbetrachtung und Implementation vorgestellt und angewandt. Daraus wurde abgeleitet, für welche Art von Anwendung Angular und React geeignet sind. Es wurden mögliche Schwierigkeiten dargestellt, die mit Verwendung der Frameworks einher gehen.

\section{Probleme}
Die Wahl, Material-UI als Design-Bibliothek zu verwenden, war in der Nachbetrachtung eine schlechte Entscheidung. Die gelieferten Komponenten haben sich doch stark unterschieden. Eine bessere Entscheidung wäre ein Framework mit Komponenten gewesen, die nur aus HTML und CSS bestehen, wie beispielsweise Bootstrap. Diese hätten dann ohne Änderungen verwendet werden können und die Implementationsdetails der Frameworks hätten noch besser hervorgehoben werden können.

Der Umstand, das dieses Thema sehr aktuelle Technologien behandelt, wirkt sich auch auf die Literatur aus. Abseits von Handbüchern, die mit einem Update veraltet sein können, gibt es kaum Fachliteratur. Die wichtigste Quelle war daher die Dokumentation der Frameworks. Ansonsten finden sich unzählige Artikel oder Videomaterial im Netz, die Qualität variiert dabei stark. 

Ein weiteres Problem war meine Einstellung zu Beginn der Implementation, den optimalen Weg zu finden und nach Best Practice zu arbeiten. Es gibt keinen allerdings nicht den einzig richtigen Best Practice, man findet immer wieder neue Lösungsansätze und Ansichten, die sich häufig auch widersprechen. Wichtig ist, Best Practices zu hinterfragen, um sie zu verstehen und auszuwählen, ob sie das vorliegende Problem lösen.

\section{Erfahrungen und Lernerfolge}
Mit Angular arbeite ich bereits aktiv seit einem Jahr, allerdings hatte ich hier die Möglichkeit Konzepte, die mir vorher nur oberflächlich bekannt waren, genauer zu betrachten und die Funktionsweise zu verstehen. Das betrifft insbesondere die Dependency Injection und Change Detection. Die Entwicklung mit Angular fällt mir dadurch leichter, auch weil ich Fehler schneller zuordnen kann.

Mit React hatte ich vor Beginn der Arbeit noch keine praktische Erfahrung, jetzt habe ich einen sehr guten Überblick über die Möglichkeiten, aber vor allem über die Grenzen der Bibliothek. In kleinen Projekten werde ich React zukünftig definitiv anwenden. Bei größeren Applikationen fällt meine Entscheidung vorerst auf Angular, da ich mit der klareren Trennung zwischen Template und Klassen strukturierter und besser arbeiten kann. 

Ich habe gelernt, wie wichtig Architekturentscheidungen in Bezug auf Webapplikationen sind und das ein Verständnis der internen Abläufe des Frameworks und der Designphilosophie der Framework-Entwickler notwendig ist. Gleichzeitig muss man die Projektanforderungen richtig analysieren und eine Architektur ableiten, die eine umständliche Problemlösung darstellt. Für React stellt sich beispielsweise die Frage, ob eine State-Management Bibliothek wie Redux wirklich notwendig ist.

\newpage
%
\begin{singlespacing}
  \printbibliography
\end{singlespacing}

\newpage
\thispagestyle{empty}
\quad
\end{document}