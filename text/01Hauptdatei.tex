%-----------------------------------------------%
%   Vorlage Abschlussarbeit HTWK Leipzig        %
%   Fakultät Maschinenbau und Energietechnik    %
%   Anpassungen/Support/Weiteres auf Anfrage    %
%                                               %
%        Der Code wird unter der Lizenz         %
%   „Creative Commons Namensnennung-Weitergabe  %
%   unter gleichen Bedingungen“ in Version 4.0  % 
%   (abgekürzt „CC-by-sa 4.0“) veröffentlicht.  %
%   _________________________________________   %
%-----------------------------------------------%

% Dokumentenklasse %
\documentclass[
ngerman,
toc=flat,
toc=chapterentrywithdots,
captions=tableabove,
listof=entryprefix,
listof=leveldown,
fontsize=12pt,
numbers=noenddot,
parskip=half,
headsepline,
footsepline,
footheight=25pt,
headheight=25pt,
usegeometry,
]{scrreprt}

% Laden der Pakete %
\usepackage{babel}
\usepackage{lmodern}
\usepackage[T1]{fontenc}

\usepackage{geometry}
\geometry{
	includeheadfoot,
	top = 20mm,
	left = 25mm,
	right = 20mm, 
	bottom = 30mm,
	bindingoffset = 5mm,
	}

\usepackage[letterspace=150]{microtype}
\usepackage[onehalfspacing]{setspace}
\BeforeStartingTOC[toc]{\singlespacing} 
\BeforeStartingTOC[lot]{\singlespacing\renewcommand\autodot{:}}
\BeforeStartingTOC[lof]{\singlespacing\renewcommand\autodot{:}}

\usepackage[
    backend=biber,
    style=numeric-comp,
    sorting=none,
    defernumbers=true
]{biblatex}
\addbibresource{Hilfsdateien/literatur.bib}
\usepackage{csquotes}

\usepackage{scrlayer-scrpage}
\clearscrheadfoot
\chead{{--~}\pagemark{~--}}

\renewcommand*{\chapterpagestyle}{scrheadings}
\RedeclareSectionCommand[%
  beforeskip=0pt
]{chapter}
\RedeclareSectionCommands[%
  beforeskip=0pt,
  afterskip=1pt plus .1\baselineskip minus .167\baselineskip
]{section,subsection}

\usepackage{graphicx} 
\usepackage{pdfpages}
\usepackage{siunitx}
\sisetup{
    locale=DE,
    per-mode=fraction,
}

\input{Hilfsdateien/06Anhangsverzeichnis}

% Platz für weitere Pakete %
%\usepackage[]{}

\usepackage[
    pdfauthor=Felix Schmeißer,
    pdfsubject=Komponentenbasierte Frontent-Frameworks - Angular und React im Vergleich,
    plainpages=false
]{hyperref}

% Beginn des Dokumentes %
\begin{document}
\newpage 
\thispagestyle{empty}
\quad
\addtocounter{page}{-1}
\newpage
%
\begin{titlepage}
{\centering
\includegraphics[]{Hilfsdateien/HTWK_logo.eps}\\[8ex]
{\large \textbf{Fakultät Informatik und Medien} \par}
{\large Studiengang Medieninformatik \par}
\vspace{1.75cm}
{\Large \textbf{Komponentenbasierte Frontend-Frameworks - Angular und React im Vergleich}\par}
\vspace{1.25cm}
{Bachelorarbeit \par}
{zur Erlangung des akademischen Grades \par} 
{\large  \textbf{Bachelor of Science} \par}
\vspace{1.25cm}
{\large vorgelegt von \par}
\vspace{1.25cm}
{Felix Schmeißer\\[3ex]
geb. am 22.\,02.\,1999\\[3ex]
in München\\[3ex]
69578 \par}}
\vfill
{\noindent Verantwortlicher Hochschullehrer: Prof. Dr. rer. nat. Klaus Hering\\
Leipzig, Juni 2020 -- September 2020}
\end{titlepage}


\thispagestyle{empty}
\begin{center}
\large \lsstyle Erklärung
\end{center}
\vspace{1.5cm}
{\doublespacing
Ich versichere wahrheitsgemäß, diese Arbeit selbständig angefertigt, alle benutzten Hilfsmittel vollständig und genau angegeben und alles kenntlich gemacht zu haben, was aus Arbeiten anderer unverändert oder mit Abänderungen entnommen wurde.\par
\vspace{2cm}
\noindent
\begin{minipage}[t]{6.5cm}
% gepunktete Linie
\dotfill \\
% Text unter der Linie
\onehalfspacing
Felix Schmeißer\\
Leipzig, den \today
\end{minipage}
\clearpage
\thispagestyle{empty}
\begin{center}
\large \lsstyle
Danksagung
\end{center}
\vspace{1.5cm}
Zunächst möchte ich meinem Betreuer Prof. Dr. rer. nat. Klaus Hering danken.

\vdots

Zuletzt danke ich meinen Freunden, meinen Eltern sowie meiner Familie für die ständige Unterstützung während meines Studiums. 
\tableofcontents
\addchap{Abbildungs- und Tabellenverzeichnis}
\listoffigures
\listoftables
%
\chapter{Einleitung}
Ein Textbeispiel. \cite{schlosser}

Ein neuer Absatz und ein weiteres Zitat \cite{nadler}.
\captionof{figure}{Bildunterschrift}
\captionof{table}{Tabellenüberschrift}
%\input{Kapitel/...}
%\input{Kapitel/...}
%\input{Kapitel/...}
%
\begin{singlespacing}
\printbibliography
\end{singlespacing}

\appendix
%\input{Kapitel/anhang}
%
\chapter{Anhangskapitel}
\captionof{figure}{Bildunterschrift im Anhang}
\captionof{table}{Tabellenüberschrift im Anhang}
\newpage
\thispagestyle{empty}
\quad
\end{document}