\chapter{Auswertung}
\section{Vergleich}
\subsubsection{Komponenten}
Komponenten in Angular bestehen aus Klassen mit einem dazugehörigen Template. Lifecycle-Methoden kann der Entwickler für bestimmte Aufgaben (z.B. Backend-Anfragen) verwenden.
Die Template Syntax erweitert HTML, um Komponenten und Direktiven verwenden zu können, auf Eigenschaften der zugrundeliegenden Komponente zuzugreifen und bei Änderung in der View aktualisieren (bidirektionale Bindung). Mitgelieferte Direktiven wie \inlinecode{ngFor} oder \inlinecode{ngIf} ermöglichen die dynamische Generierung von HTML, Pipes verarbeiten Daten auf Ebene der Templates. Mit In- und Outputs können Komponenten Daten in der Hierarchie nach unten oder oben weitergeben.

Komponenten in React bestehen aus Klassen, die eine render-Funktion bereitstellen, oder Funktionen, die JSX zurückliefern. Auf den Lifecycle kann für Klassen ebenfalls mit Methoden, für Funktionen mit React Hooks zugegriffen werden. Die Templates werden als JSX direkt in JavaScript definiert, Direktiven und Pipes gibt es nicht. Eigenschaften/Variablen der Klasse/Funktion können direkt verwendet werden. Mit Props werden Daten oder Callback-Funktionen an Kindelemente weiter gereicht, Komponenten werden bei Änderungen neu gerendert (unidirektionale Bindung).

Angular setzt auf Services, um von Komponenten unabhängige Logik abzukapseln. Sie werden per Dependency Injection bereitgestellt. React hat dafür keine Entsprechungen und überlässt das dem Entwickler.

\subsubsection{Change Detection und Performance}
Für die Change Detection verwenden beide Frameworks unterschiedliche Algorithmen, die in der Vergangenheit viele Optimierungen erhalten haben. Bezogen auf die Geschwindigkeit kann hier keine gesicherte Aussage getroffen werden, da dazu keine Benchmarks gefunden werden konnten. Häufig wird nur die initiale Ladezeit in Betracht gezogen. Der Unterschied ist dabei nicht signifikant und davon abhängig, welche zusätzlichen Frameworks mit React verwendet werden\cite{Benchmark}. In der Kochbuch-Testapplikation waren die Übergänge in der Angular-Variante spürbar flüssiger. Daraus sollte allerdings kein genereller Schluss gezogen werden, aufgrund der geringeren Erfahrung mit React könnten eventuelle Fehler die Performance beeinträchtigen.

\subsubsection{Projektarchitektur}
Angular orientiert sich an klassischen MV*-Mustern und erweitert sie mit Services. Projekte werden mit Modulen strukturiert. Komplexes State-Management kann mit Services gelöst werden, alternativ werden Frameworks wie Redux verwendet. Für viele Aufgaben gibt es eine Lösung direkt aus dem Framework: Internationalisierung, komplexe Formulare, eine integrierter HTTP Client oder Animation, nur in seltenen Fällen sind weitere Frameworks notwendig.

React hat keine Module oder Services, JSX ist per Definition gegensätzlich zu MV*Mustern. Für komplexes State-Management muss eine eigene Lösung implementiert oder ebenfalls auf externe Bibliotheken/Frameworks wie Redux zugegriffen werden. Diese Abhängigkeit von externen Lösungen gilt für die meisten Aufgaben, beispielsweise Internationalisierung (react-intl), HTTP-Clients (axios) oder Animationen (react-animations).

\subsection{Lernkurve}
Die benötigte Einarbeitungszeit ist subjektiv und abhängig von den Fähigkeiten und der Erfahrung des Entwicklers. Trotzdem lässt sich die These aufstellen, dass React an sich eine flachere Lernkurve besitzt. Der Grund liegt im deutlich kleineren Funktionsumfang von React. Es gibt keine Dependency Injection, Services, Module oder Direktiven, TypeScript ist keine Pflicht. Selbst für kleinere Angular-Projekte sollte grundsätzliches Verständnis vorliegen. Wird React mit zusätzlichen Bibliotheken als Einheit gegenüber Angular betrachtet, dann nähern sich die Lernkurven an.

\subsection{Anwendungsbereiche}
Angular ist für große Enterprise-Anwendungen geeignet, da es out-of-the-box beliebig skalierbar ist. Bemerkbar macht sich dieser Umstand in der Größe, unverpackt verbraucht Angular 566KB Speicher, React lediglich 97,5KB\footnote{siehe: https://gist.github.com/Restuta/cda69e50a853aa64912d}. Bei kleineren Projekten stellt sich dann die Frage, ob ein Framework dieser Größe notwendig ist. Die notwendige Boilerplate für Dekoratoren und Module macht einen größeren Anteil aus.

React bringt selbst nur einen Bruchteil an Lösungen mit sich, das wirkt sich entsprechend auf die Größe aus. Der Vorteil ist, dass sich React auch für kleine Projekte anbietet. Mit rein funktionalen Komponenten ohne State lassen sich in sehr kurzer Zeit dynamische Websites erstellen, durch JSX muss nicht zwischen HTML und JavaScript hin und her gesprungen werden. Unter Zuhilfenahme zusätzlichen Frameworks sind React-Projekte trotzdem beliebig skalierbar. Der Nachteil hierbei ist allerdings, dass das Projekt dadurch zusätzlichen Abhängigkeiten unterliegt, da die Pakete ebenfalls up-to-date gehalten werden müssen. Der Verwaltungsaufwand ist in diesem Bereich also höher. Bei Angular-Projekten muss in der Regel nur das Framework selbst aktualisiert werden.

Die Entscheidung für oder gegen Angular respektive React ist am Ende immer abhängig von der zu lösenden Aufgabe und den gegebenen Voraussetzungen. React ist grundsätzlich in jedem Szenario verwendbar, aber je komplexer die Aufgabe ist, desto durchdachter muss die gewählte Architektur im Zusammenspiel verschiedener Frameworks sein. Angular ist für kleinere Websites mit statischen oder nur begrenzt dynamischen Inhalten ungeeignet, weil ein Großteil der gebotenen Lösungen nicht benötigt wird. Für Unternehmen ist die Entscheidung auch vom Personal abhängig, sie müssen sich der Fähigkeiten und Erfahrungen der beteiligten Entwickler bewusst sein. Das kann im Zweifel schwerer gewichtet sein, als die Eignung der Technologie.