\chapter{Einleitung}

\section{Problemstellung}
Es gibt unzählige JavaScript-Frameworks und es kommen ständig neue hinzu. Als Entwickler muss man abwägen,
welche Lösung für das speziell vorliegende Szenario geeignet ist.

Angular ist ein umfangreiches Frontend-Framework und kann damit nahezu
jede Aufgabe in diesem Bereich abdecken. Für sehr viele Problemstellungen
gibt es eine Lösung direkt aus dem Framework. Die vorgesehene Archtitektur ist
MVC bzw. MVVM und forciert damit eine strikte Trennung, die Entwicklung in großen Teams vereinfacht.

React als JavaScript-Bibilothek ist deutlich reduzierter. Abseits der elementaren Funktion von React
werden Community-Erweiterungen verwendet. Demzufolge muss man hier zwischen verschiedenen
Lösungsmöglichkeiten abwägen. React verwendet mit JSX eine JavaScript-Erweiterung,
welche HTML und JavaScript kombiniert. Das macht die Entwicklung von Komponenten deutlich schneller,
bricht allerdings mit MV*-Architekturen.

\section{Ziel der Arbeit}
Das Ziel der Arbeit ist, Gemeinsamkeiten und Unterschiede zwischen den Technologien herauszuarbeiten,
auch mögliche Vor- und Nachteile hinsichtlich der Architektur werden angeschnitten.
Dazu wird im Rahmen der Arbeit eine Testanwendung mit den Frameworks implementiert.
Durch Steuerung der Datenmenge und künstliche Geschwindigkeitsdrosselung können verschiedene
Szenarien simuliert werden, um die Anwendungsbereiche der Frameworks einzugrenzen.

Welche Gemeinsamkeiten und Unterschiede besitzen Angular und React und welche Anwendungsbereiche
ergeben sich aus der Performance in unterschiedlichen Auslastungsszenarien?

\section{Vorgehensweise}

Zunächst ein theoretischer Vergleich beider Technologien, der die grundlegenden Features beschreibt und gegenüberstellt.
Im zweiten Schritt wird der Vergleich praktisch durchgeführt. Ziel ist es, eine identische Anwendung einmal in Angular
und React zu implementieren, um die Unterschiede zu verdeutlichen und Grenzen aufzuzeigen. Im Anschluss werden verschiedene
Testszenarien ausgewertet, um Anforderungen kleiner Anwendungen und größerer Enterprise-Produkte zu vergleichen.
Abschließend werden die Beobachtungen eingeordnet und Schlüsse über Vor- und Nachteile beider Technologien gezogen.
Zudem werden kurz Lösungsmöglichkeiten für etwaige Probleme diskutiert, damit einher geht ein Ausblick über
die Erweiterbarkeit und Einbindung von externen Lösungen.

\section{Inhaltlicher Aufbau der Arbeit}

\section{Abgrenzung}

Diese Arbeit geht auf die essentiellen Funktionalitäten beider Frameworks ein und stellt die entsprechende Umsetzung im jeweils
anderen vor. Fortgeschrittene Themen und Erweiterungen werden nicht genauer betrachtet, werden aber mit Verweis auf offizielle Quellen
in die Argumentation eingebunden.

Die Arbeit hat nicht den Anspruch, die teils kontroverse und nicht eindeutige Terminologie aufzuarbeiten. An dieser Stelle dient das Entwurfsmuster Model-View-Controller (MVC) als Beispiel. In der Literatur finden sich durchaus verschiedene Ansichten, ab wann eine gewählte Architektur dieses Entwurfsmuster erfüllt. Die Arbeit liefert für solche Begriffe eine begründete Einordnung, ohne den Anspruch zu haben, sämtliche Möglichkeiten abzubilden.

\section{Begriffe}
\subsubsection{Frameworks}
\subsubsection{MVC}