\chapter{Einleitung}
\subsubsection{Problemstellung}
Es gibt unzählige JavaScript-Frameworks und es kommen ständig neue hinzu. Als Entwickler muss abgewägt werden, welche Lösung für das speziell vorliegende Szenario geeignet ist.

Angular ist ein umfangreiches Frontend-Framework und kann damit nahezu jede Aufgabe in diesem Bereich abdecken. Für sehr viele Problemstellungen gibt es eine Lösung direkt aus dem Framework. 

React als JavaScript-Bibliothek ist deutlich reduzierter. Abseits der elementaren Funktion von React werden externe Bibliotheken oder Frameworks verwendet. Demzufolge muss hier zwischen verschiedenen Möglichkeiten entschieden werden.

\subsubsection{Ziel der Arbeit}
Das Ziel der Arbeit ist, die Technologien umfassend einzuführen und daraus Gemeinsamkeiten und Unterschiede zu erkennen. Auch werden mögliche Vor- und Nachteile hinsichtlich der Architektur angeschnitten. Dazu wird im Rahmen der Arbeit eine Testanwendung mit beiden Frameworks implementiert. Das Ziel ist, die geeigneten Anwendungsszenarien der Technologien daraus abzuleiten.

\subsubsection{Vorgehensweise}
Zunächst erfolgt eine allgemeine Einführung mit der Entstehungsgeschichte, zugrundeliegenden Technologien und dem Aufbau neuer Projekte. Im Anschluss werden die Konzepte von Angular und React beschrieben. Danach wird der Vergleich praktisch durchgeführt. Ziel ist es, eine identische Anwendung sowohl in Angular
als auch in React zu implementieren, um die Unterschiede zu verdeutlichen und Grenzen aufzuzeigen. Abschließend werden die Beobachtungen eingeordnet und Schlüsse über Vor- und Nachteile beider Technologien gezogen.

\subsubsection{Abgrenzung}
Die Arbeit gibt eine fundierte Übersicht über die Grundprinzipien von Angular und React. Eine komplette Abhandlung ist nicht möglich und für die Beantwortung der zentralen Fragestellung auch nicht notwendig.

\subsubsection{Bibliothek und Framework}
Ein Framework gibt einen Rahmen für die Entwicklung vor und bestimmt die Architektur der Anwendung. Einige Frameworks liefern außerdem fertige Komponenten und Klassen, sondern auch Laufzeitumgebungen und Interfaces zum Erstellen neuer Bestandteile, Testsysteme und Auslieferung der Anwendung als Produktionsversion.

Eine Bibliothek ist eine Sammlung von einzelnen Programmteilen wie Klassen oder Komponenten, Routinen oder ganzen Programmen, die für die Entwicklung genutzt werden können und die Arbeit unterstützen. Im Gegensatz zu Frameworks definieren Sie nicht den Rahmen der Anwendung, sondern sind eigenständige Bausteine.

Angular ist ein Webframework, React eine Bibliothek. Im Folgenden werden beide Technologien zur Vereinfachung als Framework zusammengefasst.