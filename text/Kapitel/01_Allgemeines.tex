\chapter{Allgemeines}

\section{Entstehung}

\subsection{Angular}
Die Historie von Angular beginnt 2009 als Nebenprojekt der Entwickler Miško Hevery und Adam Abrons. Hevery störte die notwendige Menge an Boilerplate für Datenbanken, Datenbankzugriff und Sicherheitsvorkehrungen. Die Idee war, diese Bestandteile zu abstrahieren und ein Framework zu entwickeln, dass von Designern ohne fortgeschrittene Programmierkenntnisse verwendet werden kann. 

\subsubsection{AngularJS}
Hevery arbeitete zu dieser Zeit bei Google mit 2 weiteren Entwicklern an Google Feedback. Die Entwicklung wurde immer schwerfälliger, woraufhin er das Projekt mit dem eigenen Framework in 3 Wochen neu schrieb und den Quelltext um 90\% reduzieren konnte. Google begann, immer mehr interne Projekte mit dem in AngularJS benannten Framework umzusetzen.

Im Oktober 2010 wurde AngularJS mit der Versionsnummer 0.9.0 als erste stabile Version auf GitHub veröffentlicht, im Juni 2012 Version 1.0.0. Das Framework war durch innovative Ansätze von Beginn an erfolgreich, dazu gehören:
\begin{itemize}
  \item Dependency Injection
  \item wiederverwendbare HTML-Bausteine
  \item zweiseitige Datenbindung (hält Model und View konsistent)
  \item Single-Page-Applikation (einzelnes HTML-Dokument lädt Inhalte dynamisch)
\end{itemize}

Die Version 1.7.0 im Mai 2018 führte als letzter Release zu vorherigen Versionen inkompatible Änderungen ein. Bis Juli 2021 wird AngularJS im Long Time Support unterstützt. Es werden lediglich Sicherheitsfehler und durch neue Versionen der gängigen Browser erzwungene Bugs behoben.\cite{NgJsHistory}\cite{NgConf2014}

\subsubsection{Angular 2}
2016 veröffentlichte Google den Nachfolger von AngularJS unter dem verkürzten Namen Angular, von Grund auf neu geschrieben und nicht abwärtskompatibel. Die neue Version beinhaltet die JavaScript-Obermenge TypeScript und verzichtet auf den Support älterer Browser. Die Template Syntax ist weiter verbessert worden, die Kapselung in Modulen und dynamisches Nachladen von Komponenten kommt hinzu. \cite{AngularWiki}

\subsection{React}
Die Entstehungsgeschichte von React ähnelt der von Angular. In 2011 standen Entwickler der Facebook Ads App vor einigen Problemen. Neue Features ließen sich mit der existierenden Codebase immer schwieriger umsetzen und eine grundlegende Überarbeitung war notwendig, um weiter entwickeln zu können. Aus diesem Umstand heraus entwickelte Jordan Walke einen ersten Prototypen namens FaxJs\footnote{siehe: https://github.com/jordwalke/FaxJs}. Dieser wurde in der folgenden Zeit auch eingesetzt und mit dem Namen React weiterentwickelt. Mit der Übernahme von Instagram durch Facebook gab es Gründe für eine Open-Source-Veröffentlichung der Bibliothek. Ende Mai 2013 stellte Walke React im Rahmen der JS ConfUS der Öffentlichkeit vor. In den folgenden Jahren verbreitete sich React weiter. Anfang 2015 begannen große Unternehmen wie Netflix und Airbnb ihre Anwendung mit dem Framework umzusetzen.\cite{ReactTimeline}

\section{Verwendete Sprachen}
In den folgenden Abschnitten werden die von Angular und React angewendeten Webtechnologien eingeführt. Die Angular Template Syntax und JSX sind spezifisch für Angular respektive React entwickelte Syntaxerweiterungen für HTML und JavaScript.

\subsection{Angular Template Syntax}
Angular führt eine spezielle Syntax ein, die den aktuellen HTML5-Standard um einige Funktionen erweitert. Das \inlinecode{script}-Tag wird mit einer Warnung ignoriert, um Injection-Angriffe zu vermeiden. Sonst sind alle bekannten Tags des Standards erlaubt. Im folgenden Abschnitt werden einige wichtige Syntaxerweiterungen vorgestellt.

\subsubsection{Interpolation und Ausdrücke}

Mit Interpolation lassen sich Template Expressions (Ausdrücke) im HTML-Template verwenden. Angular wertet die Ausdrücke aus und ersetzt sie im finalen Template durch einen String. Rechenoperationen oder Funktionsaufrufe sind erlaubt, das Ergebnis muss aber als String ausgewertet werden können.

\begin{listing}
\begin{minted}{html+ng2}
<h1>Benutzername: {{ userName }}</h1>
<!-- mit userName = 'Max' ... -->
<h1>Benutzername: Max</h1>

<span>Zufallszahl: {{ getRandomNumber() }}</span>
<!-- mit getRandomNumber -> 23 -->
<span>Zufallszahl: 23</span>
\end{minted}
\end{listing}

Die Ausdrücke können neben Interpolationen auch an HTML-Attribute gebunden werden: \mintinline[breaklines]{html+ng2}{<img [src]="imageSourceUrl">}.

Grundsätzlich ist beliebiger JavaScript-Code erlaubt. Nebeneffekte sollen vermieden werden, dadurch fallen Zuweisungen (\inlinecode{=}, \inlinecode{+=}, \inlinecode{-=,}), einige Schlüsselwörter (\inlinecode{new}, \inlinecode{typeof}, \inlinecode{instanceof}), In- und Dekrementierung (\inlinecode{++}, \inlinecode{-{}-}) und einige weitere Befehle ab dem ES2015-Standard weg. Auch Bitoperationen sind verboten, neu eingeführt werden der Pipe-Operator (\inlinecode{|}) zum Verwenden von Pipes (mehr dazu unter \ref{ssec:pipes}) und Null-Überprüfungen mit \inlinecode{!} oder \inlinecode{?}.

Funktionen können weiterhin Nebeneffekte verursachen. Die Ausdrücke stehen dabei immer im Kontext der aktuellen Komponente und greifen folglich auf Eigenschaften der verknüpften TypeScript-Klasse zu. Die Datenbindung ist einseitig von Model zu View. Mehr dazu in Kapitel 2.

\subsubsection{Anweisungen}
Das Gegenstück Template Expressions sind Template Statements (Anweisungen). Die Datenbindung verhält sich identisch. Anweisungen werden ausgeführt, wenn das korrespondierende Event aufgerufen wird, z.B. ein Klick-Event: 

\mintinline{html+ng2}{<button (click)="login()">Einloggen</button>}

Anweisungen erlauben, wie Ausdrücke, beliebiges JavaScript. Allerdings sind Nebeneffekte in diesem Fall der zentrale Punkt, um entsprechende Datenänderungen oder Navigation anzustoßen. Zuweisungen (operative wie \inlinecode{+=} oder \inlinecode{-=} ausgenommen) und Schlüsselwörter außer \inlinecode{new} sind erlaubt.
Der Kontext wird um den Kontext des Templates selbst erweitert. Anweisungen können folglich auch auf \inlinecode{\$event}-Objekte und im Template definierte Variablen zugreifen. Im folgenden Beispiel wird der Variable \inlinecode{adress} eine Referenz auf ein Input-Element zugewiesen: 

\mintinline{html+ng2}{<input #address placeholder="Adresse" />}

\subsubsection{Banana In A Box}
Datenbindung in Angular kann auch zweiseitig stattfinden: Wird die View verändert, etwa durch Eingabeevents, dann wird das Model aktualisiert. Das gilt umgekehrt für Änderungen des Models. Ausdrücke werden dazu in eckige und runde Klammern geschrieben, daher die Bezeichnung \glqq Banana In A Box\grqq. 

\mintinline{html+ng2}{<input [(ngModel)]="username">}.

\subsection{JSX}
JSX ist eine Syntaxerweiterung für JavaScript und ermöglicht die Verwendung von HTML innerhalb von JavaScript. Die HTML-Elemente werden am Ende zu vollständigem JavaScript kompiliert (Listing \ref{lst:listing1}).

\begin{listing}
\caption{Äquivalente Syntax in JSX}
\label{lst:listing1}
\begin{minted}{jsx}
const element = (
  <h1 className="hello">
    Hallo Welt!
  </h1>
);

const element = React.createElement(
  'h1',
  {className: 'hello'},
  'Hallo Welt!'
);
\end{minted}
\end{listing}

JSX wertet wie Angulars Template Syntax Ausdrücke aus, allerdings wird nur eine geschweifte Klammer um den Ausdruck gesetzt: \inlinecode{<h1>Hallo { user.name }!</h1>}. String-Literale können auch in Anführungszechen gesetzt werden. Der volle Funktionsumfang von JavaScript bleibt erhalten, somit können \inlinecode{if}-Statements und \inlinecode{for}-Schleifen verwendet werden (Listing \ref{lst:listing2})

\begin{listing}
\caption{HTML-Templates mit JavaScript}
\label{lst:listing2}
\begin{minted}{jsx}
  function greet(user) {
  if (user) {
    return <h1>Hallo {user.name}!</h1>;
  }
  return <h1>Hallo Gast!</h1>;
}
\end{minted}
\end{listing}

Möglicherweise schädliche Inhalte wie Nutzereingaben können problemlos in JSX eingebettet werden. Vor dem Rendern im React DOM werden die Ausdrücke in Strings konvertiert und spezielle Character umgewandelt (z.B. \& zu \&amp). Das verhindert XSS-Angriffe (Cross-site scripting).

\subsection{JavaScript}
JavaScript (JS) ist die Programmiersprache des Internets und wurde ursprünglich im Jahr 1995 von Brendan Eich entwickelt. Alle gängigen Browser besitzen eine Engine zum Kompilieren von JavaScript-Code, Google Chrome setzt beispielsweise auf das eigene Open-Source-Projekt V8. JavaScript wird unter der Bezeichnung ECMAScript standardisiert.

\subsubsection{Type Checking}

JavaScript ist eine dynamisch typisierte Sprache. Typen werden zur Laufzeit bestimmt und müssen nicht angegeben werden. Damit geht die Programmierung nur vermeintlich schneller, denn Typfehler werden erst zur Laufzeit erkannt, auch wenn der Code vorher kompiliert werden konnte. Statisch typisierte Sprachen fordern die Angabe von Typen und kompilieren bei Typ-Fehlern nicht. TypeScript und Flow sind 2 Varianten, JavaScript typsicher zu verwenden.

\begin{listing}
\caption{Laufzeit- und Typfehler}
\label{lst:listing3}
\begin{minted}{jsx}
// Dynamische Typisierung
var sampleNumber = 123;
splitStringCharacter(sampleNumber); // -> Laufzeit-Fehler

// Statische Typisierung
string sampleString = 123; // -> Compiler-Fehler
number sampleNumber = 123;
splitStringCharacter(sampleNumber) // -> Compiler-Fehler
\end{minted}
\end{listing}

\subsubsection{TypeScript}
Angular forciert die Nutzung des JavaScript-Supersets TypeScript (TS), ebenfalls eine Implementierung des ECMAScript-Standards, entwickelt von Microsoft. Es erweitert JavaScript mit zusätzlichen Features. JavaScript-Code ist valides TypeScript, nicht umgekehrt. TS wird mit seinem Compiler in gültigen JavaScript-Quelltext kompiliert.

Die neuen Features von TS kennen Entwickler aus der objektorientierten Programmierung. TS setzt, wie der Name schon andeutet, auf strengere Typisierung. Zwar gehen die Typings nach der Übersetzung verloren, allerdings werden Laufzeitfehler durch die vorgeschobene Kontrolle bereits bei der Entwicklung mit TypeScript reduziert.\cite{TypeScript}

\subsubsection{Flow}

React legt sich nicht auf einen Type Checker fest. In der Dokumentation wird sowohl die Installation von TypeScript, als auch Flow beschrieben. Flow ist keine eigenständige Sprache mit eigenem Compiler wie TypeScript, sondern ermöglicht Type Checks durch Annotationen. Mit dem Befehl \inlinecode{flow} werden markierte JavaScript-Dateien überprüft. Flow wird wie React von Facebook entwickelt.\cite{Flow}

\subsection{CSS}
Mit CSS werden die mit HTML semantisch strukturierten Elemente formatiert, das Styling bleibt dadurch unabhängig von den Inhalten. CSS wird vom W3C standardisiert, Version 3 seit 2000 fortlaufend weiterentwickelt. Neue Versionen gibt es nur noch für einzelne CSS-Module, nicht für den gesamten Standard.\cite{CSS4} 

Die Arbeitsweise mit CSS selbst ändert sich, anders als bei HTML und JavaScript, für React und Angular nicht. Entwickler können frei unter verschiedenen Präprozessoren und anderen Lösungen entscheiden. In HTML-Templates wird CSS mit kleinen Anpassungen auf gewohnte Art verwendet.

\subsubsection{Angular}
Angular erlaubt weiterhin die Nutzung von \inlinecode{class}-Selektoren für HTML-Elemente. Mit Class-Bindings\footnote{siehe: https://angular.io/guide/attribute-binding\#class-binding} können Strings, String-Arrays oder Objekte angegeben werden. Die gewohnte Verknüpfung von HTML-Knoten mit CSS-Klassen ist also weiter gegeben: \mintinline{html+ng2}{<Example class="first second">...</Example>}. Class-Binding kann sich auch auf einzelne Klassen beziehen: \mintinline{html+ng2}{[class.menu-active]="isMenuActive"}. Inline-Styles sind auch in Angular erlaubt und mit dem Style-Binding gibt es erweiterte Möglichkeiten. Es kann direkt auf CSS-Attribute zugreifen: \mintinline{html+ng2}{[style.width.px]="100"}. Auch auswertbare Ausdrücke sind erlaubt: \mintinline{html+ng2}{[style]="styleExpression"}.

Allerdings sind externe Stylesheets leichter zu pflegen, eine Vermischung mit Inline-Styles erschwert die Fehlersuche und bricht die Trennung von View-Content und Design.

\subsubsection{React}
In React werden Klassen mit dem className-Attribut zugewiesen:

\mintinline{jsx}{<span className="headline-big highlighted">Headline</span>}

Listing \ref{lst:listing4} zeigt, wie Styles in JSX auch dynamisch hinzugefügt werden können. Das \inlinecode{style}-Attribut bleibt erlaubt, allerdings gilt hier ebenfalls, dass solche Inline-Styles vermieden und eigene Stylesheets verwendet werden sollten\footnote{siehe: https://reactjs.org/docs/dom-elements.html\#style}.\cite{StylingReact}

\begin{listing}
\caption{Dynamisches Styling mit JSX}
\label{lst:listing4}
\begin{minted}{jsx}
render() {
  let className = 'button-save';
  if (!this.props.hasPermission) {
    className += 'button-deactivated';
  }
  return <button className={className}>Speichern</button>
}
\end{minted}
\end{listing}

\subsubsection{Sass}
Sass vereinfacht CSS und fügt Funktionalitäten hinzu, die die Les- und Wartbarkeit der Stylesheets verbessern, beispielsweise mit Mixins: CSS-Blöcke können als Variablen definiert und an anderen Stellen verwendet werden. Sass vereinfacht die Selektierung dadurch, dass sich Selektoren schachteln lassen. Sass wird als Präpozessor\footnote{siehe: https://developer.mozilla.org/de/docs/Glossary/CSS\_Praeprozessor} bezeichnet, weil ein Compiler die erweiterte Syntax zu nativem CSS verarbeitet. Neben Sass gibt es noch weitere Präprozessoren wie LESS, Stylus oder PostCSS.

\section{Projektaufbau}
\subsection{Projekterstellung}
\subsubsection{Angular CLI}
Die Erstellung eines Angular Projektes erfolgt mit dem Angular CLI (Command Line Interface), welches mit dem Node Package Manager (NPM) installiert wird: 

\inlinecode{npm install -g @angular/cli}

Das CLI verfügt über Befehle, die z.B. externe Bibliotheken hinzufügen:

\inlinecode{ng add <Bibliothek>}

Neue Bestandteile (z.B. Components) anlegen:

\inlinecode{ng generate <Schema>} 

Oder Tests starten:

\inlinecode{ng test}

Mit \inlinecode{ng new <Projektname>} wird ein neues Projekt in einem entsprechend benannten Ordner angelegt. Das Projekt ist hier schon startbereit, mit \inlinecode{ng serve --open} kann die Anwendung im Development Modus gestartet werden. Durch das Flag wird automatisch der Browser geöffnet (http://localhost:4200/.). Änderungen im Projekt werden erkannt und die Anwendung sofort aktualisiert.\cite{CLI}

\subsubsection{React Toolchains}
React benötigt kein umfangreiches Setup. Die einfachste Möglichkeit der Einbindung besteht darin, React innerhalb des \inlinecode{<script>}-Tags zu verknüpfen. Damit lässt sich React zu bestehenden Projekten hinzufügen, um etwa eine schrittweise Migration zu ermöglichen. Für neue Projekte, die von Beginn an mit React entwickelt werden sollen, bietet sich eine Toolchain für das entsprechende Anwendungsszenario an. Damit wird die Umgebung für die React-Entwicklung optimiert: Häufig wird bereits eine Grundstruktur und Scripte zum Starten der Anwendung mit erweiterten Debugoptionen generiert, ähnlich dem Setup einer Angular-Applikation. Die einfachste Toolchain stammt ebenfalls von Facebook und nennt sich \textit{create-react-app}. Mit dem Befehl \inlinecode{npx create-react-app <Projektname>} wie beim Angular CLI ein Projektordner generiert. Durch \inlinecode{npm start} wird die Applikation gehostet (http://localhost:3000/).\cite{CreateReactApp}


\subsection{Dateistruktur}
Die Frameworks haben nach der Erstellung mit den eben beschriebenen Tools einen ähnlichen Aufbau. Neben einem \textit{src}-Order, der die Ressourcen der Anwendung beinhaltet, gibt es einige Konfigurationsdateien für JavaScript respektive TypeScript, außerdem eine Ordner (\textit{node\_modules}), der die eingebundenen Pakete beinhaltet. Das sind zunächst die Core-Funktionalitäten der Frameworks, gegebenenfalls kommen weitere Bibliotheken dazu (Angular\footnote{siehe: https://angular.io/guide/file-structure}, React\footnote{https://www.pluralsight.com/guides/file-structure-react-applications-created-create-react-app})

Der Entwicklungsprozess beginnt bei beiden Frameworks mit der Root-Komponente, die beim Start eingebunden wird. Dort beginnt die Komponentenstruktur des Projektes. Für Angular ist das \inlinecode{app.component.ts} und das dazugehörige \inlinecode{app.module.ts}, bei React \inlinecode{App.jsx}.

Die Dateistruktur von Angular und React Anwendungen ist direkt abhängig von der Komponentenstruktur. Der Entwickler muss in der Lage sein, gesuchte Dateien schnell zu finden. Eine ordentliche Struktur ist nicht relevant für das Endprodukt, kann jedoch die Entwicklungsgeschwindigkeit beeinträchtigen. Für den jeweiligen Use-Case finden sich im Internet Lösungen, die React-Dokumentation\footnote{siehe: https://reactjs.org/docs/faq-structure.html} bietet einen grundsätzlichen Ansatz, der auch für Angular gelten kann.
